\documentclass[11pt,a4paper]{book}
\usepackage[utf8]{inputenc}
\usepackage[portuguese]{babel}
\usepackage[T1]{fontenc}
\usepackage{amsmath}
\usepackage{amsfonts}
\usepackage{amssymb}
\author{Brandon Sanderson \\ traduzido por: Guilherme Dias}
\title{Warbreaker}
\begin{document}
\maketitle
\newpage
Para a Emily, que disse sim
\tableofcontents
\chapter{Agradecimentos}
Trabalhar no Warbreaker foi um processo fora do normal de algumas maneiras; podes ler sobre isso melhor no meu website. Vale a pena dizer que eu tive um grupo de leitores alfa mais variados do que o normal, muitos deles conheço-os principalmete pelas suas contribuições nos meus forums. Tentei fazer um apanhado dos nomes de todos aqui, mas tenho a certeza que vai faltar algum. Se és um deles, envia-me um e-mail e a gente tenta colocar-te nas próximas impressões.

O primeiro agradecimento é dedicado à minha amável esposa, Emily Sanderson, com quem casei enquanto escrevia este livro. Este foi o primeiro romance que ela influenciou muito ao dar-me as suas opiniões e sugestões, a qual a sua ajuda é muito apreciada. E, como sempre, o meu agente Joshua Bilmes, e o meu editor, Moshe Feder, tiveram um enorme trabalho neste manuscrito, levando-o desde o Segundo ou Terceiro Aprimoramento até pelo menos o Oitavo.

Na Tor, muitas pessoas fizeram muito mais do que lhes era pedido. A primeira é Dot Lin, minha publicitária, com quem tem sido particularmente fixe trabalhar. Obrigado, Dot! E, como sempre, os incansáveis esforços do Larry Yoder merecem destaque, assim como o excelente trabalho do genial diretor de arte da Tor, Irene Gallo. Dan Dos Santos fez a arte da capa deste livro, e eu sugiro vivamente que passem pelo seu website e os seus outros trabalhos, porque creio que é um dos melhores no mercado neste momento. E também, Paul Stevens merece umas palavras de agradecimento por ser o elo de ligação entre a editora e os meus livros.

No departamento dos agradecimentos especiais temos Joevans3, e Dreamking47, Louise Simard, Jeff Creer, Megan Kauffman, thelsdj, Megan Hutchins, Izzy Whiting, Janci Olds, Drew Olds, Karla Bennion, Eric James Stone, Dan Wells, Isaac Stewart, Ben Olsen, Greyhound, Demented Yam, D.Demille, Loryn, Kuntry Bumpken, Vadia, U-boat, Tjaeden, Dragon Fly, pterath, BarbaraJ, Shir Hasirim, Digitalbias, Spink Longfellow, amyface, Richard “Capitão Goradel” Gordon , Swiggly, Dawn Cawley, Drerio, David B, Mi’chelle Trammel, Matthew R Carlin, Ollie Tabooger, John Palmer, Henrik Nyh, e o incansável Peter Ahlstrom.

\newpage
\begin{Huge}
Warbreaker
\end{Huge}
\newpage
\newpage 

\chapter{Prelúdio}

\textit{Engraçado}, pensou Vasher, \textit{como muitas coisas começam comigo a ser mandado para a prisão}

Os guardas riram um para o outro, cerrando a porta da cela com um estrondo. Vasher levantou-se e sacudiu-se, rodou os ombros e estremeceu. Enquanto a parte de baixo da sua cela era de madeira, a parte de cima estava apenas vedada, e ele conseguia ver três guardas a abrir a sua sacola à procura pelos seus pertences.

Um deles notou que ele estava a observar. O guarda era um gigante de uma besta de homem de cabeça rapada e uniforme sujo que mal retinha o amarelo claro e azul da guarda de T’Telir.

 
\textit{Cores Vivas}, pensou Vasher. \textit{Terei que me habituar a elas de novo}. Em qualquer nação, azuis e amarelos vibrantes seriam ridículos em soldados. Isto, porém, é Hallandren: a terra de deuses Retornados, servos Não-vivos, pesquisa Bio-Cromática e - claro - cor.

O grande guarda  deambulou até à porta da cela, deixando os seus amigos a delirarem-se com os pertences do Vasher. 

- Dizem que és duro de roer - Disse o homem tomando medidas ao Vasher.

Vasher não respondeu.

- O barista diz que dás cabo duns vinte homens numa rixa. - Disse o guarda coçando o queixo - Não me pareces tão duro assim. De qualquer forma, devias tomar melhores decisões que bater num sacerdote. Os outros, passam uma noite presos. Já tu... vais para a forca. Parvo Incolor.

 Vasher virou-se. A sua cela era funcional, para não dizer nada original. Uma fina fresta ao topo da parede deixava entrar luz, as paredes de pedra escorriam àgua e musgo, e um farto de palha sujo decompunha-se no canto.
 
 - 'Tás a ignorar-me? - o guarda perguntou, aproximando-se da cela. As cores do seu uniforme ficaram mais vivas, como se uma luz mais forte lhe incidisse. Foi uma mudança leve. Vasher não tinha tanto Fôlego restante, e por isso a sua aura não influenciava tanto as cores à sua volta. O guarda não notou a mudança da cor - assim como não tinha notado no bar, quando ele e os seus amiguinhos levantaram Vasher do chão e lançaram-no na sua carroça. Claro, a mudança era tão ligeira à vista desarmada que era quase impossível percebê-la.
 
- Ora, ora - disse um dos homens que revistava a maleta do Vasher. - Que temos aqui? - Vasher sempre achou interessante como os homens que vigiavam as masmorras tendiam a ser tão maus, ou piores, que os homens que eles guardavam. Talvez seja de propósito. A sociedade não parecia querer saber se estes homens estavam dentro ou fora das celas, desde que fossem mantidos longe de homens mais honestos.

Assumindo que tal coisa existe.

De dentro do saco do Vasher, um guarda retirou um objeto comprido enrolado em linho branco. O homem assobiou enquanto desenrolava o tecido, revelando uma comprida e fina espada numa bainha prateada. O punho era de um puro negro. 
- De quem acham que \textit{isto} foi roubado?
 
O guarda principal fitou Vasher, provavelmente a perguntar-se se Vasher era algum tipo de nobre. Apesar de Hallendren não ter aristocracia, muitos reinos vizinhos tinham os seus senhores e damas. No entanto que senhor vestiria um manto castanho pardo esfarrapado? Que senhor ostentaria nódoas negras de uma luta de bar, uma barba por fazer, e botas gastas de anos a andar? O guarda virou-se, aparentemente convencido de que Vasher não era lorde algum.

Ele estava certo. E ele estava errado.

- Deixa-me ver isso, - o guarda principal disse, tomando a espada. Ele grunhiu, claramente surpreso pelo seu peso. Virou a espada, apercebendo-se da proteção que prendia a bainha ao punho, impedindo que a espada fosse desembainhada. Ele removeu a proteção.

As cores à volta agravaram. Não ficaram mais vivas - não da forma que as vestes do guarda ficaram quando se aproximou de Vasher. Ao invés disso, elas ficaram mais \textit{fortes}. Escuras. Vermelhos tornaram-se castanhos. Amarelos endureciam para dourados. Os azuis aprofundavam.

- Cuidade, amigo - Disse Vasher gentilmente - essa espada consegue ser perigosa.

O guarda retirou os olhos da espada. Tudo estava parado. Depois o guarda bufou e afastou-se da cela do Vasher, ainda carregando a espada. Os outros dois seguiram-no, levando o saco do Vasher, entrando na sala dos guardas ao final do corredor. 

Cerrou-se a porta. Vasher imediatamente ajoelhou-se ao lado do farto de palha, selecionando um punhado dos melhores galhos. Tirou alguns fios do seu manto - estava a começar a desgastar-se na parte de baixo - e atou a palha para que ganhasse forma de uma pessoa pequena, talvez do tamanho de três polegadas, os seus membros eram como arbustos. Ele arrancou um pelo das suas sobrancelhas, colocou a contra a cabeça da figura de palha e tirou um lenço vermelho vivo do interior da sua bota.

 E depois Vasher Expirou.
 
 Fluiu-lhe do corpo, soprando para o ar, translúcido porém radiante, como a cor do óleo na água ao sol. Vasher sentiu a sair-lhe: Fôlego BioCromático chamavam-lhe os académicos. A maioria das pessoas chamava-lhe de Fôlego, apenas. Cada pessoa tem um. Ou, pelo menos, é assim que costuma ser. Uma pessoa, um Fôlego.
 
 Vasher tinha cerca de cinquenta Fôlegos, apenas os suficientes para atingir o Primeiro Aprimoramento. Ter tão poucos sabia-lhe a pouco comparando com os que ele já teve, mas muita gente consideraria cinquenta Fôlegos uma fortuna. Infelizmente, até Acordar uma figura pequena de material orgânico - usando um pouco do seu corpo como foco - consumia cerca de metade do seu Fôlego. 
 
 A pequena figura de palha estremeceu, sugando o Fôlego. Na mão do Vasher, metade do lenço vermelho vivo esmaeceu para cinzento. Vasher inclinou-se - imaginando o que queria que a figura fizesse - e completou o último passo do ritual dando a Instrução.
 
 - Busca as chaves - disse.
 
 A pequena figura levantou-se e franziu a única a sua única sobrancelha a Vasher. Vasher apontou para a sala dos guardas. De lá, ouviam-se gritos de surpresa.
 
 \textit{Não resta muito}, pensou ele.
 
 A figura de palha correu pelo chão, saltou, passando por entre as barras. Vasher removeu o manto e colocou-o no chão. Era o feitio perfeteito de uma pessoa marcada com cortes que se pareciam com as cicatrizes no corpo de Vasher, o capuz cortado com buracos a vazer dos seus olhos. Quanto mais o objeto era parecido com a forma e feitio, menos Fôlego era necessário para Despertar. 
 
 Vasher agachou, tentando não relembrar-se dos tempos em que tinha Fôlego suficiente para Acordar sem se preocupar com os feitios ou com foco. Eram outros tempos. Encolhendo-se, puxou por um tufo de cabelo da sua cabeça, espalhando-o pelo capuz do manto.
 
 Mais uma vez, Expirou.
 
 Tirou-lhe o resto do seu Fôlego. Sem isso - o manto a tremer, o cachecol perdendo o resto da sua cor -  Vasher sentiu-se... mais apagado. Perder o Fôlego não era fatal. De facto, o Fôlego extra que Vasher usava pertenceram a outras pessoas. Vasher não sabia quem eram; ele não tinha recolhido os Fôlegos ele próprio. Tinham lhe sido dados. Mas, claro, era assim que devia ser sempre. Uma pessoa não podia tirar Fôlegos à força. 
 
 Estar devoto de Fôlego mudáva-o.  As cores não pareciam tão fortes. Ele não conseguia sentir a multidão a mover-se lá em cima na cidade, uma conexão que normalmente tomava por garantido. Era a consciência que qualquer homem tem de outros - aquilo que nos dá um aviso, quando se está sonolento, quando alguém entra no quarto. No Vasher, esse sentido tinha sido aumentado cinquenta vezes. 
 
 E agora tinha ido embora. Sugado pelo manto e pelo homenzinho de palha, dando-lhes poder.
 
 O manto estremeceu. Vasher inclinou-se.
 
 - Proteje-me - comandou ele, e o manto aquietou. Levantou-se, colocando-se de volta aos ombros.
 
 A figura de palha voltou à sua janela. Trazia um conjunto de chaves grande. Os pés da figura de palha estavam manchados de vermelho. Para Vasher o vermelho carmesim do sangue parecia tão mortiço, agora.
 
 Ele tomou as chaves.
 
 - Obrigado - disse ele. Ele agradecia-lhes sempre. Não sabia porquê, considerando particularmente o que fazia a seguir. - O teu Fôlego para o meu - comandou, tocando no peito do homenzinho de palha. O homenzinho de palha caiu imediatamente da porta - a vida a ser-lhe tomada - e Vasher recuperou o seu Fôlego de volta. O sentido de consciência voltou, o conhecimento de conexão, de pertença. Ele só podia recuperar o Fôlego porque tinha sido ele próprio a Despertar esta criatura - claro, Despertar desta forma era raramente permanente. Ele usava o seu Fôlego como uma reserva, dotando-o\footnote{Dar um dote a algo/alguém} e depois recuperando-o.

 Comparando com o que ele já teve, vinte cinco Fôlegos era um número ridiculamente pequeno. Mas, comparando com nada, parecia uma infinidade. Ele arrepiou de satisfação.
 
 Os gritos da sala dos guardas cessou. A masmorra quietou. Ele tinha que se mexer.
 
 Vasher passou o braço pelas barras, usando as chaves para abrir a cela. Empurrou a porta grossa, apressando-se pelo corredor, deixando a figura de palha descartada no chão. Ele não se dirigiu à sala dos guardas - e para a saída para lá dela- mas virou-se ao invés para sul, penetrando mais a fundo masmorra dentro.

 Esta a parte mais incerta do seu plano. Encontrar uma taberna frequentada por sacerdotes dos Tons Iridescentes foi fácil o suficiente. Envolver-se numa luta de bar - e depois atacar um desses mesmos sacerdotes - tinha sido igualmente simples. Hallendren levava a suas figuras religiosas muito seriamente, e Vasher tinha ganho não a prisão habitual, numa cadeia local, mas uma viagem às masmorras do Deus Rei.

 Conhecendo o tipo de homem que costuma guardar tais masmorras, ele teve uma bela ideia de que tentariam desembainhar a Nightblood\footnote{Não encontrei adaptação plausível}. E isso criou a distração que ele necessitava para conseguir as chaves.

 Mas agora vinha a parte imprevisível.

 Vasher parou, de manto Desperto roçando. Era fácil de localizar a cela que queria, porque à sua volta havia um grande bocado de pedra cuja cor tinha sido drenada, deixando tanto paredes e portas num cinzento baço. Era um sítio para encarcerar um Despertador, não existir cor significa não poder Despertar. Vasher aproximou-se da porta, olhando pelas barras. Um homem estava suspenso pelos braços ao teto, nu e acorrentado. A sua cor era vibrante aos olhos de Vasher, a sua pele um puro bronze, as suas nódoas negras brilhantes salpicos de azul e violeta.

 O homem estava amordaçado. Outra precaução. Para Despertar, o homem precisaria de três coisas: Fôlego, cor e um Comando\footnote{Ordem?}. Os harmónicos e matizes, chamam-lhes alguns. Os Tons Iridescentes, a relação entre cor e som. Um Comando tinha que ser ditado clara e firmemente na língua materna - qualquer exitação, ou pronúncia incorreta, invalidaria o Despertar. O Fôlego seria consumido, mas o objeto não conseguiria agir.
 
 Vasher usou as chaves da prisão para destrancar a porta e depois entrou. A aura daquele homem fazia as cores muito mais vivas em comparação quando alguém se aproximava dele. Qualquer um seria capaz de reparar numa aura tão forte, apesar de ser muito mais fácil para alguém que tinha alcançado o Primeiro Aprimoramento.
 
 Não era a aura BioCromática mais forte que Vasher já viu - essas pertenciam aos Retornados, tidos como deuses aqui em Hallendren. Ainda assim, o BioCrôma do prisioneiro era impressionante e muito, muito mais forte que o próprio do Vasher. O prisioneiro carregava muitos Fôlegos. Centenas e centenas deles.

 O homem baloiçou nas correntes, estudando Vasher, os lábios amordaçados sangrando da falta de àgua. Vasher hesitou um momento e depois alcançou a mordaça e puxou-a.
 
 - Tu - sussurou o prisioneiro, tossindo ligeiramente - Estás aqui para me libertar?
 
 - Não, Vahr - disse Vasher baixinho. - Estou aqui para matar-te.
 
 Vahr retorquiu. Estar cativo não lhe fez bem. Na última vez que Vasher viu Vahr, este estava gordo. Pelo seu corpo escanzelado, ele não tinha comida havia algum tempo. Os cortes, as nódoas negras, e queimaduras na sua pele estavam recentes.
 
 Tanto a tortura como o ar aterrorizado nas profundas olheiras de Vahr contavam uma verdade importante. Fôlego só podia ser transferido por uma Comando voluntário e intencional. Porém, esse Comando podia ser encorajado.
 
 - Então, - roncou Vahr - julgas-me, como toda a gente.
 
 - A tua rebelião fracassada não é me diz respeito. Eu quero apenas o teu Fôlego.
 
 - Tu e toda a corte de Hallendren.
 
 - Sem. Mas tu não o vais dar a um Retornado. Irás dá-lo a mim. Uma troca pela tua morte.
 
 - Não parece lá muito uma troca. - Havia uma dureza - uma falta de emoção - em Vahr que Vasher não tinha reparado na útlima vez que se despediram, anos antes. 
 
 \textit{Estranho}, pensou Vasher, \textit{que eu finalmente, passado todo este tempo, encontro algo no homem com o qual me consigo identificar}.
 

(metade da pag 20)

Vasher kept a wary distance from Vahr. Now that the man’s voice was free, he could Command. However, he was touching nothing except for the metal chains, and metal was very difficult to Awaken. It had never been alive, and it was far from the form of a man. Even during the height of his power, Vasher himself had only managed to Awaken metal on a few, select occasions. Of course, some extremely powerful Awakeners could bring ob- jects to life that they weren’t touching, but that were in the sound of their voice. That, however, required the Ninth Heightening. Even Vahr didn’t have that much Breath. In fact, Vasher knew of only one living person who did: the God King himself.

That meant Vasher was probably safe. Vahr contained a great wealth of
\end{document}