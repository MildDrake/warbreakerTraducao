\documentclass[11pt,a4paper]{book}
\usepackage[utf8]{inputenc}
\usepackage[portuguese]{babel}
\usepackage[T1]{fontenc}
\usepackage{amsmath}
\usepackage{amsfonts}
\usepackage{amssymb}
\author{Brandon Sanderson \\ traduzido por: Guilherme Dias}
\title{Warbreaker}
\begin{document}
\maketitle
\newpage
Para a Emily, que disse sim
\tableofcontents
\chapter{Agradecimentos}
Trabalhar no Warbreaker foi um processo um fora do normal de algumas maneiras; podes ler sobre isso melhor no meu website. Vale a pena dizer que eu tive um grupo de leitores alfa mais variados do que o normal, muitos deles conheço-os principalmete pelas suas contribuições nos meus forums. Tentei fazer um apanhado dos nomes de todos aqui, mas tenho a certeza que vai faltar algum. Se és um deles, envia-me um e-mail e a gente tenta colocar-te nas próximas impressões.

O primeiro agradecimento é dedicado à minha amável esposa, Emily Sanderson, com quem casei enquanto escrevia este livro. Este foi o primeiro romance que ela influenciou muito ao dar-me as suas opiniões e sugestões, ao qual  a sua ajuda é muito apreciada. E, como sempre, o meu agente Joshua Bilmes, e o meu editor, Moshe Feder, tiveram um enorme trabalho neste manuscrito, levando-o desde o Segundo ou Terceiro Realce até pelo menos o Oitavo.

Na Tor, muitas pessoas fizeram muito mais do que lhes era pedido. A primeira é Dot Lin, minha publicitária, com quem tem sido particularmente fixe trabalhar. Obrigado, Dot! E, como sempre, os incansáveis esforços do Larry Yoder merecem destaque, assim como o excelente trabalho do genial diretor de arte da Tor, Irene Gallo. Dan Dos Santos fez a arte da capa deste livro, e eu sugiro vivamente que passem pelo seu website e os seus outros trabalhos, porque creio que é um dos melhores no mercado neste momento. E também, Paul Stevens merece umas palavras de agradecimento por ser o elo de ligação entre a editora e os meus livros.

No departamento dos agradecimentos especiais temos Joevans3, e Dreamking47, Louise Simard, Jeff Creer, Megan Kauffman, thelsdj, Megan Hutchins, Izzy Whiting, Janci Olds, Drew Olds, Karla Bennion, Eric James Stone, Dan Wells, Isaac Stewart, Ben Olsen, Greyhound, Demented Yam, D.Demille, Loryn, Kuntry Bumpken, Vadia, U-boat, Tjaeden, Dragon Fly, pterath, BarbaraJ, Shir Hasirim, Digitalbias, Spink Longfellow, amyface, Richard “Capitão Goradel” Gordon , Swiggly, Dawn Cawley, Drerio, David B, Mi’chelle Trammel, Matthew R Carlin, Ollie Tabooger, John Palmer, Henrik Nyh, e o incansável Peter Ahlstrom.

\newpage
\begin{Huge}
Warbreaker
\end{Huge}
\newpage
\newpage

\chapter{Prelúdio}

\textit{Engraçado}, pensou Vasher, \textit{como muitas coisas começam comigo a ser mandado para a prisão}

Os guardas riram um para o outro, cerrando a porta da cela com um estrondo. Vasher levantou-se e sacudiu-se, rodou os ombros e estremeceu. Enquanto a parte de baixo da sua cela era de madeira, a parte de cima estava apenas vedada, e ele conseguia ver três guardas a abrir a sua sacola à procura pelos seus pertences.

Um deles notou que ele estava a observar. O guarda era um gigante de uma besta de homem de cabeça rapada e uniforme sujo que mal retinha o amarelo claro e azul da guarda de T’Telir.

 
\textit{Cores Vivas}, pensou Vasher. \textit{Terei que me habituar a elas de novo}. Em qualquer nação, azuis e amarelos vibrantes seriam ridículos em soldados. Isto, porém, é Hallandren: a terra de deuses Retornados, servos Não-vivos, pesquisa Bio-Cromática e - claro - cor.

O grande guarda  deambulou até à porta da cela, deixando os seus amigos a delirarem-se com os pertences do Vasher. 

- Dizem que és duro de roer - Disse o homem tomando medidas ao Vasher.

Vasher não respondeu.

- O barista diz que dás cabo duns vinte homens numa rixa. - Disse o guarda coçando o queixo - Não me pareces tão duro assim. De qualquer forma, devias tomar melhores decisões que bater num sacerdote. Os outros, passam uma noite presos. Já tu... vais para a forca. Parvo Incolor.

 Vasher virou-se. A sua cela era funcional, para não dizer nada original. Uma fina fresta ao topo da parede deixava entrar luz, as paredes de pedra escorriam àgua e musgo, e um farto de palha sujo decompunha-se no canto.
 
 - 'Tás a ignorar-me? - o guarda perguntou, aproximando-se da cela. As cores do seu uniforme ficaram mais vivas, como se uma luz mais forte lhe incidisse. Foi uma mudança leve. Vasher não tinha tanto Fôlego restante, e por isso a sua aura não influenciava tanto as cores à sua volta. O guarda não notou a mudança da cor - assim como não tinha notado no bar, quando ele e os seus amiguinhos levantaram Vasher do chão e lançaram-no na sua carroça. Claro, a mudança era tão ligeira à vista desarmada que era quase impossível percebê-la.
 
- Ora, ora - disse um dos homens que revistava a maleta do Vasher. - Que temos aqui? - Vasher sempre achou interessante como os homens que vigiavam as masmorras tendiam a ser tão maus, ou piores, que os homens que eles guardavam. Talvez seja de propósito. A sociedade não parecia querer saber se estes homens estavam dentro ou fora das celas, desde que fossem mantidos longe de homens mais honestos.

Assumindo que tal coisa existe.

De dentro do saco do Vasher, um guarda retirou um objeto comprido enrolado em linho branco. O homem assobiou enquanto desenrolava o tecido, revelando uma comprida e fina espada numa bainha prateada. O punho era de um puro negro. 
- De quem acham que \textit{isto} foi roubado?
 
O guarda principal fitou Vasher, provavelmente a perguntar-se se Vasher era algum tipo de nobre. Apesar de Hallendren não ter aristocracia, muitos reinos vizinhos tinham os seus senhores e damas. No entanto que senhor vestiria um manto castanho pardo esfarrapado? Que senhor ostentaria nódoas negras de uma luta de bar, uma barba por fazer, e botas gastas de anos a andar? O guarda virou-se, aparentemente convencido de que Vasher não era lorde algum.

Ele estava certo. E ele estava errado.

- Deixa-me ver isso, - o guarda principal disse, tomando a espada. Ele grunhiu, claramente surpreso pelo seu peso. Virou a espada, apercebendo-se da proteção que prendia a bainha ao punho, impedindo que a espada fosse desembainhada. Ele removeu a proteção.

As cores à volta agravaram. Não ficaram mais vivas - não da forma que as vestes do guarda ficaram quando se aproximou de Vasher. Ao invés disso, elas ficaram mais \textit{fortes}. Escuras. Vermelhos tornaram-se castanhos. Amarelos endureciam para dourados. Os azuis aprofundavam.

- Cuidade, amigo - Disse Vasher gentilmente - essa espada consegue ser perigosa.

O guarda retirou os olhos da espada. Tudo estava parado. Depois o guarda bufou e afastou-se da cela do Vasher, ainda carregando a espada. Os outros dois seguiram-no, levando o saco do Vasher, entrando na sala dos guardas ao final do corredor. 

Cerrou-se a porta. Vasher imediatamente ajoelhou-se ao lado do farto de palha, selecionando um punhado dos melhores galhos. Tirou alguns fios do seu manto - estava a começar a desgastar-se na parte de baixo - e atou a palha para que ganhasse forma de uma pessoa pequena, talvez do tamanho de três polegadas, os seus membros eram como arbustos. Ele arrancou um pelo das suas sobrancelhas, colocou a contra a cabeça da figura de palha e tirou um lenço vermelho brilhante do interior da sua bota.

 E depois Vasher Respirou.
 
 Fluiu-lhe do corpo, soprando para o ar, translúcido porém radiante, como a cor do óleo na água ao sol. Vasher sentiu a sair-lhe: Fôlego BioCromático chamavam-lhe os académicos. A maioria das pessoas chamava-lhe de Fôlego, apenas. Cada pessoa tem um. Ou, pelo menos, era assim que costuma ser. Uma pessoa, um Fôlego.
 
 Vasher  















\end{document}